\documentclass{article}
\pagestyle{headings} % for page numbering

%\pagestyle{headings}
\usepackage[dvips]{graphics,color}
\usepackage{amsfonts}
\usepackage{amssymb}
\usepackage{amsmath}
\usepackage{latexsym}
\usepackage{framed}
\usepackage{enumitem}

\usepackage[square,numbers]{natbib}
\bibliographystyle{abbrvnat}

\usepackage{hyperref}

\usepackage[disable]{todonotes}

\newcommand{\algoname}[1]{\textnormal{\textsc{#1}}}

\newcommand{\nizkpok}{\algoname{NiZKPoK}}
\newcommand{\gen}{\algoname{Gen}}
\newcommand{\sign}{\algoname{Sign}}
\newcommand{\verify}{\algoname{Verify}}
\newcommand{\link}{\algoname{Link}}

\newcommand{\sidx}{\ensuremath{*}}

\newcommand{\pkset}{\ensuremath{X}}

\newcommand{\ix}{\ensuremath{x_i}}
\newcommand{\sx}{\ensuremath{x_\sidx}}

\newcommand{\ipk}{\ensuremath{X_i}}
\newcommand{\spk}{\ensuremath{X_\sidx}}

\newcommand{\iy}{\ensuremath{y_i}}
\newcommand{\sy}{\ensuremath{y_\sidx}}
\newcommand{\oy}{\ensuremath{y^\prime}}
\newcommand{\oiy}{\ensuremath{y^\prime_i}}

\newcommand{\iC}{\ensuremath{C_i}}
\newcommand{\sC}{\ensuremath{C_\sidx}}
\newcommand{\oC}{\ensuremath{C^\prime}}
\newcommand{\oiC}{\ensuremath{C^\prime_i}}

\newcommand{\iv}{\ensuremath{v_i}}
\newcommand{\sv}{\ensuremath{v_\sidx}}
\newcommand{\ov}{\ensuremath{v^\prime}}
\newcommand{\oiv}{\ensuremath{v^\prime_i}}

\newcommand{\setI}{\ensuremath{\mathcal{I}}}

\newcommand{\alg}{\ensuremath{\mathcal{A}}}

\newcommand{\hs}{\ensuremath{\mathcal{H}_S}}
\newcommand{\hp}{\ensuremath{\mathcal{H}_P}}

\newcommand{\qed}{\hfill\ensuremath{\square}}

\newcommand{\curve}{\ensuremath{E(\mathbb{F}_q)}}

\begin{document}

\title{One-Time, Zero-Sum Ring Signature \\ {\small WORKING DRAFT}}
\author{Conner Fromknecht}
\date{\today}

\maketitle

\section{Introduction}
  
Is Bitcoin a currency?  The jurisdictions of institutions all over the world
have culminated in everything but a unanimous decision.  In practice, Bitcoin
offers the majority of features present in typical fiat currencies such as
payments of arbitrary amount and exchanges for another currency.  However,
Bitcoin fails to meet one of the key properties of a true currency\textemdash
fungibility.

Not all Bitcoins are created equal.  Since the transaction history is public, so
too are the balances and payments made by its participants.  After any
transaction is completed, the receiver is able to view the complete history of
the coin. If the coin's history includes transactions that are known to be a
part of a scandal, the receiver can choose to reject the payment.  This
inherently makes some Bitcoin less valuable than others, since accepting tainted
coins assumes the risk that the next receiver may not accept them.  By
definition, this inequality between coins prevents Bitcoin from being fungible.

The goal of this work is to design a more fungible cryptocurrency.  In order to
do so, we tackle two distinct problems that allow a third party to discriminate
against the coins used in a standard Bitcoin transaction.  The first is lack of
anonymity, which allows an individual to trace the history of a transaction and
determine the accounts that have held the coin before.  This work builds upon
much of CryptoNote's \cite{S13} anonymous transaction scheme to ensure that the
sender and receiver are obfuscated, preventing third parties from explicitly
backwards constructing a path to the sender.  

However, this alone is not enough.  Since the transaction amounts are still
visible, this allows any two transactions of equal amounts to be possibly
linked, and thus jeopardizing anonymity.  Therefore, our solution must also
hide transaction amounts.  This work begins with a similar approach to the one
outlined by Gregory Maxwell's Confidential Transaction scheme \cite{M15}.  This
allows the sender to prove that the value is within a certain range, say
$[0,2^l)$.  In order to publicly verify this, the signer must coordinate the
blinding factors and amounts of different inputs such that they result in a zero
sum.  However, when using CryptoNote's ring signatures, the signer has no
control over the blinding factors of other inputs.  Thus, we propose a new ring
signature construction, called a One-Time, Zero-Sum Ring Signature (OZRS) that
proves the output amount is equal to exactly one of the committed input values.
It also proves that a commitment receives a new blinding factor after each
transaction and allows only the recipient to learn the amount enclosed.
Furthermore, the size of an OZRS is the same as the One-Time Ring Signatures
used in the standard CryptoNote protocol.

By combining the unlinkability and confidentiality properties of this work, the
receiver of a transaction is able to verify its amount but also learns
substantially less about the coin's history.  Since this is true for any future
recipient, the receiver can accept the payment as is with greater certainty that
it will not be rejected by another party.

Lastly, it has been brought to my attention that a similar protocol had been
developed by MRL \cite{N15} prior to the beginning of this work.  Though the
ideas captured in this work are entirely of my own conclusion, it is worth
noting the technical achievements and additional extensions their solution
proposes make it a superior alternative.

\section{Construction}

Building on CryptoNote's architecture, this work requires relatively simple
changes to the high level protocol.  The modifications include adding a single
field to a transaction output and replacing the One-Time Ring Signature with an
OZRS.  Furthermore, every transaction amount is committed using a Pedersen
Commitment and accompanied by a range proof using the Borromean scheme described
in \cite{MP15}.  For simplicity, we assume that the range proof is done in
binary, but the results can be extended to any publicly known encoding.

\subsection{Transactions}

Here we describe how to construct a single output transaction, where some user
is trying to send the value {\ov} to the standard address $(A=aG, B=bG)$.  These
steps operate in addition to the unmodified CryptoNote protocol, unless
otherwise specified here.

When building a single output transaction, the signer also chooses a random
number $q \in Z_N$ and adds the {\em blind seed} $Q=qG$ to the transaction
output.  The signer then computes $\oy = \hs(qB)$ which is called the {\em
output blinding factor}, where $B$ is taken to be the receiver's public key.
Using $\oy$, the output commitment $\oC = \oy G + \ov H$ is constructed by first
deterministically generating blinding factors $\gamma^{(i)}$ to commit each of
the $l$ bits $\beta_i$ in {\ov}.  These blinding factors are computed by
\begin{align*}
  \gamma^{(1)} &= \hs(\oy) \\
  \gamma^{(i+1)} &= 
    \begin{cases}
      \oy - \sum_{j=1}^{i} \gamma^{(j)} &: i=l \\
      \hs(\gamma^{(i)}) &: otherwise.
    \end{cases}
\end{align*}
The signer then outputs the final $\oC = \sum_{i=1}^l c_i$, where $c_i =
\gamma^{(i)} G + \beta_i H$ and $\beta_i$ is either $2^i$ or 0 depending on the
$i^{th}$ bit of {\ov}.  Each $c_i$ and $\gamma^{(i)}$ is further used to
construct the range proof of $\oC$ using the techniques in \cite{MP15}.

A receiver uses the blind seed to compute $\oy = \hs(bQ)$ where $b$ is taken to
be the receiver's super secret private key.  A receiver with knowledge of $\oy$
is also able to recover each of the $\gamma^{(i)}$ blinding factors.  Then, he
can recover the bits of {\ov} by checking $c_i - \gamma^{(i)} G = 0$ or $d_i -
\gamma^{(i)} G = 0$, where the challenge public key $d_i = c_i - 2^i H$.  If the
first is true, then $\beta_i=0$; if the second check passes then $\beta_i=2^i$.
Note that the signer of a transaction could pick $Q^\prime \neq qG$, which
results in different $\gamma^{(i)}$ with high probability.  The receiver can
easily detect this event if both $c_i$ and $d_i$ failed the above test.  In this
case, the receiver should not accept the transaction as payment, since he will
be unable to spend it.

\subsubsection{Multiple Outputs}

Supporting transactions with $m$ outputs only requires a small modification to
the single output case.  We compute one $\oiC = \oiy G + \oiv H$ for each output
amount as described above, providing a range proof for each.  We then compute
$\oC = \sum_{i=1}^m \oiC$ and $\oy = \sum_{i=1}^m \oiy$.  Note that each output
now has its own blind seed, so they can each be recovered independently.
Lastly, instead of creating a single transaction public key $R=rG$, we create
one for each output.  This allows the outputs of a transaction to be spent
independently of each other, much like how they are in Bitcoin.

\subsubsection{Transaction Structure}

This section describes the format of a transaction incorporating the above
changes.  The construction of the OZRS is described in the subsequent section.
\begin{align*} 
  & \text{\bf{INPUT}}       &&                                           \\%&& \\
  %& \text{Key Image: } && I = \sx \hp(\sx)                                \\%&& [33 \text{ bytes}] \\
  & \text{Input Transaction Hashes: } && \{ \hs(T_i) \}_n                 \\%&& [32n \text{ bytes}] \\
  & \text{\bf{OUTPUT}}     &&                                            \\%&& \\
  & \text{Transaction Public Keys: } && \{ R_i = r_i G \}_m               \\%&& [33m \text{ bytes}] \\
  & \text{Destination Keys: } && \{ P_i = \hs(r_i A_i)G + B_i \}_m        \\%&& [33m \text{ bytes}] \\
  & \text{Blind Seeds: } && \{ Q_i = q_i G \}_m                           \\%&& [33m \text{ bytes}] \\
  & \text{Commitments: } && \{ \oiC = \oiy G + \oiv H \}_m                \\%&& [33m \text{ bytes}] \\
  & \text{Range Proofs: } && \{ \pi_l(\oiC) \}_m                          \\%&& [(33l + 32(2l+1))m \text{ bytes}] \\
  & \text{Fee } &&                                                       \\%&& [8 bytes] \\
  & \text{\bf{SIGNATURE}} &&                                              \\%&& \\
  & \text{OZRS: } && \Pi = (I, e, r_1, \dots, r_n, s_1, \dots, s_n)       \\%&& [32(2n+1) \text{ bytes}] 
\end{align*} 

\subsection{One-Time, Zero-Sum Ring Signature}

For any transaction with $n$ inputs and $m$ outputs, let $\pkset = \{ \ipk = \ix
G\}_{i \in [1,n]}$ be the set of input destination keys and $\sidx \in [1,n]$ to
be the index of signer's public key {\spk}.  Furthermore, let $C = \{ {\iC} =
{\iy} G + \iv H \}_{i \in [1,n]}$ be the set input commitments, where each $\iC$
commits each {\ipk} to the value {\iv}.  We call each {\iy} an input blinding
factor.  After constructing a transaction, the signer also holds the new output
blinding factor {\oy} and total output value {\ov} in
\[ \oC = \oy G + \ov H = \sum_{i=1}^{m} \oiC + fee \cdot H, \]
where each $\oiC$ represents an individual output commitment.  Here we present
a ring signature formulation constructed as an AOS ring signature \cite{AOS02}
that uses a three-way chameleon hash to prove the following properties:
\begin{enumerate}
  \item The signer knows at least one secret key {\ix} for a public key {\ipk}

  \item The signer knows the secret key {\sx} corresponding to the preimage $I =
  {\sx} \hp({\spk})$ of {\spk}.

  \item The sum of the output commitments {\oC} holds a value equal to the
  sender's input {\sC}.
\end{enumerate}
More formally, the ring signature is a Non-Interactive, Zero-Knowledge Proof of
Knowledge on a message $M$ such that all values other than {\sx}, {\sy}, and
{\oy} are known to the prover, defined 
\begin{align*}
\nizkpok[M](\sx, \sy, \oy) : \{ \exists i : 
& ~ \oC - \iC = (\oy - \iy)G + 0 \cdot H\\
& \land \spk = \ix G \\
& \land I = \sx \hp(\ipk)\}.
\end{align*}

The One-Time, Zero-Sum Ring Signature scheme consists of the four operations
({\gen}, {\sign}, {\verify}, {\link}).  
\begin{itemize}
  \item ${\gen}(N, G) \rightarrow (a, A)$

    Choose $a \leftarrow Z_N$ at random. \\
    Compute $A=aG$ and output $(a, A)$.

\ \\

  \item ${\sign}(M, C, {\oC}, {\oy}, {\sy}, {\sx}, X) \rightarrow \Pi $

    Compute the signing key's preimage and commitment differences
    \begin{align}
      I   &= {\sx} \hp({\spk}) \\
      D_i &= {\oC} - {\iC} = ({\oy} - {\iy}) G + (a^\prime - a_i)H.
    \end{align}
    Next, we build the non-interactive challenge $e$.  Choose $k_1, k_2
    \leftarrow Z_N^2$ at random.  Starting at index $*+1$, compute \[
    e_{*+1}^{(1)} = \hs(M ~ || ~ k_1 G ~ || ~ k_2 G ~ || ~ k_2 \hp(\spk) )\]
    Continue computing successive $e_i^{(\cdot)}$, wrapping around after $i=n$,
    until $e_{*}^{(2)}$ using the following steps 
    \begin{align} 
      r_i, s_i &\leftarrow Z_N^2 \\
      e_i^{(2)} &= \hs(e_i^{(1)}) \\
      e_{i+1}^{(1)} &= \hs(M ~ || ~ r_iG - e_i^{(1)}D_i ~ || ~ s_iG - e_i^{(2)}{\ipk} ~ || ~ s_i \hp({\ipk}) - e_i^{(2)}I)
    \end{align}
    Lastly, we set $r_*, s_*$ so that the hash hits $e_{*+1}^{(1)}$
    by
    \begin{align*}
      r_* &= k_1 + e_*^{(1)}({\oy} - {\sy}) \\
      s_* &= k_2 + e_*^{(2)}{\sx}
    \end{align*}
    Assign $e=e_1^{(1)}$ and output the final proof 
    $\Pi = (I, e, r_1, \dots, r_n, s_1, \dots, s_n)$.

\ \\

  \item ${\verify}(\Pi, M, C, {\oC}, X) \rightarrow \{0,1\} $

    First compute each commit difference $D_i$ using equation (2).  Starting
    with $e=e_1^{(1)}$, compute the forward $e_i^{(\cdot)}$ for $i \in [1, n]$
    using relations (4) and (5).  The verifier then checks that 
    \[e = \hs(M ~ 
    || ~ r_nG - e_n^{(1)}D_n ~ 
    || ~ s_nG - e_n^{(2)}X_n ~ 
    || ~ s_n \hp(X_n) - e_n^{(2)}I)\]
    and ${\link}(I)$ fails.  If both of these are met, return 1. Otherwise, return
    0.

\ \\

  \item $\link(I) \rightarrow \{0,1\}$

  Let {\setI} be the set of all spent preimages.  Return 1 if $I \in {\setI}$,
  otherwise return 0.
\end{itemize}

\section{Security}

Below we present the security proofs for the OZRS scheme.  These proofs are
similar to those in the original CryptoNote paper \cite{S13}, with the added
property divisibility, which is demonstrated in a manner that maintains the
confidentiality of the committed inputs.

The security properties are:
\begin{itemize}
  \item {\bf Linkability:} Given the secret keys $\{\ix\}_n$ and each
  corresponding public key {\ipk}, it is impossible to produce $n+1$ valid
  signatures $\{\Pi_j\}_{n+1}$ such that all of them have different preimages
  $I_j$.

  \item {\bf Exculpability:} Given the public keys $X=\{\ipk\}_n$, at most
  $n-1$ secret keys {\ix} excluding $i = *$, and the preimage $I_*=\sx
  \hs(\spk)$, it is impossible to produce a valid signature $\Pi$ with $I_*$.

  \item {\bf Unforgeability:} Given the public keys $X=\{\ipk\}_n$, it is
  impossible to produce a valid signature $\Pi$.
  
  \item {\bf Anonymity:} Given a signature $\Pi$ and its corresponding public
  keys $X=\{\ipk\}_n$, it is impossible to determine the index $*$ of the signing
  key with probability $p > \frac{1}{n}$.

  \item {\bf Divisibility:} Given the set of input commitments $C=\{\iC\}_n$
  each committing to the value $a_i$ and the sum of the output commitments {\oC}
  committing to $a^\prime$, it is impossible to create a signature $\Pi$ such
  that $a^\prime \neq a_*$.  This implies that each transaction perfectly
  divides (or is equivalent to) the value of signer's input commitment.

\end{itemize}

\subsection{Linkability}

{\bf Theorem 1:} OZRS is linkable under the random oracle model. \\

{\it Proof.} For contradiction, assume that an adversary can produce $n+1$ valid
signatures $\{\Pi_i\}_{n+1}$ such that $I_i \neq I_j$ for any $i,j \in
[1,\dots,n+1]$.  Since $|X| = n$, there must be at least one $I^\prime \neq \ix
\hp( \ipk )$ for any $i \in [1,\dots,n]$. Consider the corresponding signature
$\Pi^\prime = (I^\prime, e, r_1, \dots, r_n, s_1, \dots, s_n)$.  Since all $n+1$
signatures are valid, $\verify(\Pi^\prime)=1$, implying
\begin{align*}
  U_i^\prime &= r_i G - e_i^{(1)} D_i \\
  V_i^\prime &= s_i G - e_i^{(2)} \ipk \\
  W_i^\prime &= s_i \hp(\ipk) - e_i^{(2)} I^\prime \\
  e_{i+1}^{(1)} &= \hs( M ~ || ~ U_i^\prime ~ || ~ V_i^\prime ~ || ~ W_i^\prime).
\end{align*}
The second and third equalities imply
\begin{align*}
  \log_G V_i^\prime &= s_i - e_i^{(2)} \ix \\
  \log_{\hp(\ipk)} W_i^\prime &= s_i - e_i^{(2)} \log_{\hp(\ipk)} I^\prime.
\end{align*}
Let $x^\prime = \log_{\hp(\ipk)} I^\prime$, known to the signer.  Assume WLOG
that the signature is computed using $*$ as the signing index.  To begin the
forward computation, the signer has committed to some $k_2^\prime$ and
$k_2^{\prime\prime}$ in
\begin{align*}
  V_* &= k_2^\prime G \\
  W_* &= k_2^{\prime\prime} \hp(\spk).
\end{align*}
Since the signature verifies, it must be that $V_* = V_*^\prime$ and $W_* =
W_*^\prime$, implying
\begin{align*}
  k_2^\prime &= s_* - e_*^{(2)} \sx \\
  k_2^{\prime\prime} &= s_* - e_*^{(2)} x^\prime.
\end{align*}
Combining the above equations provides the following relation
\[
  k_2^{\prime\prime} = k_2^\prime + e_*^{(2)}(\sx - x^\prime).
\]
Since $\nexists i : x_i = x^\prime$, the signer knew $e_*^{(2)}$ before
beginning the forward computation.  This yields a contradiction, since this
requires finding a preimage of {\hs}, which succeeds with only negligible
probability.  Therefore, OZRS is linkable under the random oracle model. \qed

\subsection{Exculpability}

{\bf Theorem 2:} OZRS is exculpable under the discrete logarithm assumption in
the random oracle model. \\

{\it Proof.} For contradiction, assume that an adversary can produce a valid
signature $\Pi=(I, e, r_1, \dots, r_n, s_1, \dots, s_n)$ with $I = \sx
\hp(\spk)$ given $\{\ix\}$ for $i \in [1,\dots,n] \setminus \{*\}$.  Then, we
can construct an algorithm {\alg} which solves the discrete logarithm in
{\curve}.

Suppose $(G, P) \in \curve^2$ is a given instance of the DLP where the goal is
recover $s$ such that $G = sP$.  Using the techniques in \cite{HOLYGRAIL},
{\alg} simulates the random and signing oracles and makes the adversary produce
two valid signatures with $P = \spk \in X$: 
\[
\Pi = (I,e,r_1,\dots,r_n,s_1,\dots,s_n) 
\text{~and~} 
\Pi^\prime = (I,e^\prime,r_1^\prime,\dots,r_n^\prime,s_1^\prime,\dots,s_n^\prime).
\]

Since $I = \sx \hp(\spk)$ in both signatures, {\alg} computes $\sx =
\log_{\hp(\spk)} I = \frac{s_* - s_*^\prime}{e_*^{(2)} - e_*^{(2)\prime}} \mod
N$. {\alg} outputs $\sx$ since $V_* = s_* G + e_*^{(2)} \spk = s_*^\prime G +
e_*^{(2)\prime} \spk$ and $\spk = P$.

\subsection{Unforgeability}

{\bf Theorem 3:} If OZRS is linkable and exculpable, then it is unforgeable. \\

{\it Proof.}  For contradiction, assume that an adversary can forge a signature
for given set of public keys $X$: $\Pi_0 = (I_0,\dots)$.  Consider all valid
signatures (produced by honest signers) for the same message $M$ and $X$:
$\Pi_1, \Pi_2, \dots, \Pi_n$.  There are two possible cases:
\begin{enumerate}
  \item $I_0 \in \{I_i\}_{i=1}^n$.  Which contradicts exculpability.
  \item $I_0 \notin \{I_i\}_{i=1}^n$.  Which contradicts linkability.
\end{enumerate}

\subsection{Anonymity}

{\bf Theorem 4:} OZRS is exculpable under the decisional Diffie-Hellman
assumption in the random oracle model.

{\it Proof.} To be completed.

\subsection{Divisibility}

{\bf Theorem 5:} OZRS is divisible under the random oracle model.

{\it Proof.} To be completed.

\bibliography{bibliography}

\end{document}

